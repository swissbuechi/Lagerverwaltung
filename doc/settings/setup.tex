\usepackage[utf8]{inputenc}
\usepackage[ngerman]{babel}
\usepackage[nottoc,numbib]{tocbibind}
\usepackage{graphicx}
\graphicspath{ {images/} }

\usepackage{tabularx}
\usepackage{tikz}

\usepackage{url}

\usepackage{fancyhdr}
\usepackage{geometry}

\usepackage{listings}
\usepackage{xcolor}

\usepackage{afterpage}

\definecolor{codegreen}{rgb}{0,0.6,0}
\definecolor{codegray}{rgb}{0.5,0.5,0.5}
\definecolor{codepurple}{rgb}{0.58,0,0.82}
\definecolor{backcolour}{rgb}{0.95,0.95,0.92}

\lstdefinestyle{mystyle}{
    backgroundcolor=\color{backcolour},   
    commentstyle=\color{codegreen},
    keywordstyle=\color{magenta},
    numberstyle=\tiny\color{codegray},
    stringstyle=\color{codepurple},
    basicstyle=\ttfamily\footnotesize,
    breakatwhitespace=false,         
    breaklines=true,                 
    captionpos=b,                    
    keepspaces=true,                 
    numbers=left,                    
    numbersep=5pt,                  
    showspaces=false,                
    showstringspaces=false,
    showtabs=false,                  
    tabsize=2
}

\lstset{style=mystyle}

\geometry{
  footskip=10mm,	% Position Fusszeile (Abstand zum Text)
  headsep=10mm,	% Position Kopfzeile (Abstand zum Text)
  top=25mm,
  right=20mm,
  bottom=20mm,
  left=40mm,
  footnotesep=10mm	% Abstand Text - Fussnote
}
% \setlength{\headheight}{10mm}

\usepackage[T1]{fontenc}
\renewcommand{\rmdefault}{phv} % Arial
\renewcommand{\sfdefault}{phv} % Arial

\hyphenation{Rund-funk-ge-neh-mi-gung} % Eigene breaks in Wörtern

\RequirePackage{url} % Escape durch \url{}
\RequirePackage[hidelinks]{hyperref} % Referenzen erzeugen (klickbare URLs)
\hypersetup{breaklinks=true} % URLs dürfen brechen
\def\UrlBreaks{\do\/\do-} % URLs dürfen auch bei Zeichen brechen

\RequirePackage[printonlyused]{acronym} % Acronym

\setcounter{tocdepth}{2} % Max tiefe des Inhaltsverzeichnis           
\setcounter{secnumdepth}{2} % Max tiefe der Sections

\usepackage[autostyle=true, german=swiss]{csquotes}

% Formatting-------------------------------------------------------------------
% LTeX: enabled=false

\usepackage[onehalfspacing]{setspace}% Zeilenabstand von 1.5

\setlength\parindent{0pt}% Remove paragraph indent
\usepackage[skip=12pt]{parskip}% Add space between paragraphs

% % Chapter title formatting
\RequirePackage{titlesec}
\titleformat{name=\chapter}{}{}{0em}{\bf\LARGE\thechapter \hspace{1.15cm}}
\titleformat{name=\chapter,numberless}{}{}{0em}{\bf\LARGE}

\titlespacing{\chapter}      {0pt}{0pt}{12pt}

% Section title formatting
\makeatletter
\renewcommand*{\@seccntformat}[1]{\hbox to 1.5cm{\csname the#1\endcsname}}
\makeatother

\titlespacing{\section}      {0px}{24pt}{12pt}
\titlespacing{\subsection}   {0px}{24pt}{12pt}
\titlespacing{\subsubsection}{0px}{24pt}{12pt}

% Header formatting
\pagestyle{fancy}
\fancyhf{}
% \lhead{\myTitle}
% \rhead{\thepage}
\chead{\thepage}
\renewcommand{\headrulewidth}{0pt} % Remove header line

% Chapter header formatting
\fancypagestyle{plain}{
  \fancyhf{}
  % \lhead{\myTitle}
  % \rhead{\thepage}
  \chead{\thepage}
  \renewcommand{\headrulewidth}{0pt} % Remove header line
}

% Fügt im Literaturverzeichnis "Verfügbar unter" und optional "Letzter Zugriff" 
% zu .bib entries mit url und urldate hinzu.
% \DeclareFieldFormat{formaturl}{Verfügbar unter #1}
% \DeclareFieldFormat{formatdate}{Letzter Zugriff: #1}

% \renewbibmacro*{url+urldate}{%
% 	% \iffieldundef{urlyear}{%
% 		% \printtext[formaturl]{\printfield{url}}\nopunct%
% 	% }
%   {%
%   \printtext[formatdate]{\printurldate}%
% 	\printtext[formaturl]{\printfield{url}}\adddot\space%
% 	}%
% }

% Caption formatting
\usepackage{caption}
\captionsetup{justification=raggedright,singlelinecheck=false}

% Source for figures
\newcommand{\source}[1]{\vspace{-10pt} \caption*{\small Quelle: {#1}} }

% Numbering of figures
\counterwithout{figure}{chapter}